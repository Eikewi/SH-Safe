

    \begin{tcolorbox}
    Die Zielbestimmungen dienen dazu, die Anforderungen nach Priorität zu klassifizieren. Die Einteilung erfolgt in \textit{Muss}-, \textit{Soll}- und \textit{Kann}-Kriterien sowie \textit{Abgrenzungskriterien}, um den Projektumfang klar zu definieren.
    \end{tcolorbox}
    
    \section{Muss-Kriterien}
    Die folgenden Anforderungen sind essenziell für die Funktionsfähigkeit der Anwendung und müssen ohne Kompromisse umgesetzt werden:
    \begin{itemize}
        \item Das Smartphone notifiziert den User wenn er einem Unfall-Hotspot zu nahe kommt.
        \item Der User bestätigt einen Unfall-Hotspot
        \item Der User meldet unbekannte Unfall-Hotspots
        \item Implementierung einer Admin-Weboberfläche.
        \item Der User soll in der Lage sein Informationen an den Admin zu verschicken
        
    \end{itemize}
    
    \section{Soll-Kriterien}
    Diese Funktionen sind erwünscht und sollen nach Möglichkeit implementiert werden:
    \begin{itemize}
        \item Implementierung eines User-Login
        \item Die User sollen die Möglichkeit sich via Kommentare und Voting über die Unfall-Hotspots auszutauschen 
        \item Einzeichung einer Heat-Map in die Google Maps Maps Oberfläche.
     
    \end{itemize}
    
    \section{Kann-Kriterien}
    Diese Funktionen sind optional und werden nur umgesetzt, falls Zeit und Ressourcen es erlauben:
    \begin{itemize}
       \item Implementierung eines Forums.
    \end{itemize}
    
    \section{Abgrenzungskriterien}
    Folgende Anforderungen werden explizit nicht umgesetzt:
    \begin{itemize}
        \item Zahlungs- oder Monetarisierungsfunktionen innerhalb der Plattform.
    \end{itemize}
