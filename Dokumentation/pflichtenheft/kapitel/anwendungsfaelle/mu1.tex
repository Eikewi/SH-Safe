\begin{figure}[h]
    \centering
    \begin{tabularx}{\textwidth}{X|X}
        \textbf{Anwendungsfall ID} & MU-1 \\ \hline
        \textbf{Anwendungsfallname} & Location als false positive Melden\\ \hline
        \textbf{Initiierender Akteur} & Anwender \\ \hline
        \textbf{Weitere Akteure} & - \\ \hline
        \textbf{Kurzbeschreibung} & 
        Der Anwender hat die GPS-Daten eines Unfall-Hotspotes entdeckt, welche nicht der Realität korrektheit. \\ \hline
        \textbf{Vorbedingungen} & 
        \\ \hline
        \textbf{Nachbedingungen} & 
        Der Request wurde an den Admin gemeldet. \\ \hline
        \textbf{Ablauf} &
            \begin{enumerate}
                \item Der Anwender befindet sich auf der Startseite der APP und klickt auf den Button "Melden".
                \item Das System bietet den Nutzer den aktuellen Standort als false Positive zu melden oder eine GPS-Location selber zu wählen.
            \end{enumerate}
        \\ \hline
        \textbf{Alternative} & - \\ \hline
        \textbf{Ausnahme} & - \\ \hline
        \textbf{Benutzte Anwendungsfälle} & - \\ \hline
        \textbf{Spezielle Anforderungen} & - \\ \hline
        \textbf{Annahmen} & - 
    \end{tabularx}
    \caption{Anwendungsfall MU-1 (Location als false positive Melden)}
    \label{fig:anwendungsfall-app-tabelle-mz-1}
\end{figure}