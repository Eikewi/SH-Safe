\begin{figure}[h]
    \centering
    \begin{tabularx}{\textwidth}{X|X}
        \textbf{Anwendungsfall ID} & MA-1 \\ \hline
        \textbf{Anwendungsfallname} & Admin Login \\ \hline
        \textbf{Initiierender Akteur} & Admin \\ \hline
        \textbf{Weitere Akteure} & / \\ \hline
        \textbf{Kurzbeschreibung} & 
        Der Admin loggt sich mit seinen spezifischen Daten ein. \\ \hline
        \textbf{Vorbedingungen} & 
        Admin-Zugang existiert. \\ \hline
        \textbf{Nachbedingungen} & 
        Admin ist authentifiziert und kann Admin-Funktionen nutzen. \\ \hline
        \textbf{Ablauf} &
            \begin{enumerate}
                \item Admin klickt auf „Admin Login“.
                \item Admin gibt Zugangsdaten ein.
                \item System prüft die Rechte.
                \item Admin erhält spezifische Rechte.
            \end{enumerate}
        \\ \hline
        \textbf{Alternative} &
            \begin{enumerate}
                \item Falsche Login-Daten: Fehlermeldung.
            \end{enumerate}
        \\ \hline
        \textbf{Ausnahme} & / \\ \hline
        \textbf{Benutzte Anwendungsfälle} & / \\ \hline
        \textbf{Spezielle Anforderungen} & / \\ \hline
        \textbf{Annahmen} & /
    \end{tabularx}
    \caption{Anwendungsfall MA-1 (Admin Login)}
    \label{fig:anwendungsfall-app-tabelle-ma-1}
\end{figure}