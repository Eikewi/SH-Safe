\begin{figure}[h]
    \centering
    \begin{tabularx}{\textwidth}{X|X}
        \textbf{Anwendungsfall ID} & MA-4 \\ \hline
        \textbf{Anwendungsfallname} & Anfragen verwalten \\ \hline
        \textbf{Initiierender Akteur} & Admin \\ \hline
        \textbf{Weitere Akteure} & Nutzer \\ \hline
        \textbf{Kurzbeschreibung} & 
        Der Admin sieht alle offenen, vom Nutzer gestellten Anfragen und entscheidet darüber. \\ \hline
        \textbf{Vorbedingungen} & 
        Admin ist eingeloggt. \\ \hline
        \textbf{Nachbedingungen} & 
        Die betroffene Anfrage ist akzeptiert oder abgelehnt; das System ist ggf. angepasst. \\ \hline
        \textbf{Ablauf} &
            \begin{enumerate}
                \item (Use-Case „Admin Login“) muss zuvor erfolgt sein.
                \item Admin öffnet „Anfragen verwalten“.
                \item Admin genehmigt oder lehnt einzelne Anfragen ab.
                \item System ändert ggf. die Daten.
            \end{enumerate}
        \\ \hline
        \textbf{Alternative} & 
        Keine Anfragen vorhanden: System bleibt auf dem aktuellen Stand. \\ \hline
        \textbf{Ausnahme} & / \\ \hline
        \textbf{Benutzte Anwendungsfälle} & Admin Login \\ \hline
        \textbf{Spezielle Anforderungen} & 
        Es sollte mindestens eine Anfrage vom Nutzer existieren. \\ \hline
        \textbf{Annahmen} & /
    \end{tabularx}
    \caption{Anwendungsfall MA-4 (Anfragen verwalten)}
    \label{fig:anwendungsfall-app-tabelle-ma-4}
\end{figure}