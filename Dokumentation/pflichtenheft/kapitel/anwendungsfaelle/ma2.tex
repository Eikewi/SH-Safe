\begin{figure}[h]
    \centering
    \begin{tabularx}{\textwidth}{X|X}
        \textbf{Anwendungsfall ID} & MA-2 \\ \hline
        \textbf{Anwendungsfallname} & GPS-Daten verwalten \\ \hline
        \textbf{Initiierender Akteur} & Admin \\ \hline
        \textbf{Weitere Akteure} & / \\ \hline
        \textbf{Kurzbeschreibung} & 
        Der Admin verwaltet über seine Web-Oberfläche die GPS-Daten. Hier wählt er ganz spezifisch GPS-Daten aus Requests aus oder er verwaltet nach eigenem Ermessen \\ \hline
        \textbf{Vorbedingungen} & 
        Admin ist eingeloggt. \\ \hline
        \textbf{Nachbedingungen} & 
        Die GPS-Daten wurden von dem Admin bearbeitet.\\ \hline
        \textbf{Ablauf} &
            \begin{enumerate}
                \item (Use-Case „Admin Login“) muss zuvor erfolgt sein.
                \item Admin öffnet „Einstellungen“.
                \item Admin wählt „GPS-Daten verwalten“.
                \item Admin ändert Parameter.
                \item System übernimmt die Parameter.
            \end{enumerate}
        \\ \hline
        \textbf{Alternative} & / \\ \hline
        \textbf{Ausnahme} & / \\ \hline
        \textbf{Benutzte Anwendungsfälle} & Admin Login \\ \hline
        \textbf{Spezielle Anforderungen} & / \\ \hline
        \textbf{Annahmen} & /
    \end{tabularx}
    \caption{Anwendungsfall MA-2 (GPS-Daten verwalten)}
    \label{fig:anwendungsfall-app-tabelle-ma-2}
\end{figure}