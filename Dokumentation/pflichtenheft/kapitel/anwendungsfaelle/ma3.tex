\begin{figure}[h]
    \centering
    \begin{tabularx}{\textwidth}{X|X}
        \textbf{Anwendungsfall ID} & MA-3 \\ \hline
        \textbf{Anwendungsfallname} & Nutzer verwalten \\ \hline
        \textbf{Initiierender Akteur} & Admin \\ \hline
        \textbf{Weitere Akteure} & / \\ \hline
        \textbf{Kurzbeschreibung} & 
        Der Admin kann Nutzer hinzufügen oder entfernen. \\ \hline
        \textbf{Vorbedingungen} & 
        Der Admin hat sich als Admin eingeloggt. \\ \hline
        \textbf{Nachbedingungen} & 
        Nutzerkonten sind entsprechend angepasst. \\ \hline
        \textbf{Ablauf} &
            \begin{enumerate}
                \item (Use-Case „Admin Login“) muss zuvor erfolgt sein.
                \item Admin öffnet Menüpunkt „Nutzer verwalten“.
                \item Admin legt neue Nutzer an oder ändert/löscht bestehende Nutzer.
            \end{enumerate}
        \\ \hline
        \textbf{Alternative} & / \\ \hline
        \textbf{Ausnahme} & / \\ \hline
        \textbf{Benutzte Anwendungsfälle} & Admin Login \\ \hline
        \textbf{Spezielle Anforderungen} & / \\ \hline
        \textbf{Annahmen} & /
    \end{tabularx}
    \caption{Anwendungsfall MA-3 (Nutzer verwalten)}
    \label{fig:anwendungsfall-app-tabelle-ma-3}
\end{figure}