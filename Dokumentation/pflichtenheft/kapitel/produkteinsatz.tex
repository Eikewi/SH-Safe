\section{Anwendungsgebiet}
Die App KielSafe soll im Alltag der Verkehrsteilnehmenden helfen, Unfälle im Straßenverkehr zu vermeiden. 
Dazu werden vorhandene Unfalldaten ausgewertet, um Stellen mit einer auffälligen Häufung von Unfällen 
im Straßennetz von Kiel und der näheren Umgebung zu erkennen und auf einer Karte darzustellen. KielSafe wird als mobile Anwendung auf Smartphones genutzt. Anhand der aktuellen Position der Nutzenden per GPS wird geprüft, ob sie sich in der Nähe einer bekannten Unfallstelle befinden. Wenn dies der Fall ist, 
erhalten sie einen Hinweis direkt auf dem Mobiltelefon, zum Beispiel in Form einer kurzen Nachricht oder 
eines Hinweises auf der Karte. So sollen kritische Stellen bewusster wahrgenommen und die Aufmerksamkeit 
im Straßenverkehr erhöht werden. Die App versteht sich als Ergänzung zu bestehenden Maßnahmen der Verkehrssicherheitsarbeit. Sie ersetzt 
weder Verkehrszeichen noch bauliche Maßnahmen, sondern unterstützt die Nutzenden dabei, ihr eigenes 
Verhalten an bekannten Gefahrenschwerpunkten anzupassen und vorausschauender zu fahren.

\newpage

\section{Zielgruppe}
Die Applikation richtet sich generell an Verkehrsteilnehmende, die sich aktiv im Straßenverkehr bewegen und ein 
mobiles Endgerät mitführen. Im Fokus stehen dabei folgende Zielgruppen:

\begin{description}
    \item[Fahranfänger:] Personen mit geringer Fahrpraxis (z.\,B. Führerscheinneulinge, Fahrende im Rahmen von 
    begleitetem Fahren ab 17 Jahren). Diese Zielgruppe soll durch frühzeitige Warnungen vor Unfallhotspots dabei 
    unterstützt werden, Gefahrenschwerpunkte bewusster wahrzunehmen und sicherere Fahrstrategien zu entwickeln.

    \item[Berufspendler im Pkw:] Personen, die regelmäßig mit dem Auto auf wiederkehrenden Strecken (z.\,B. zwischen 
    Wohn- und Arbeitsort) unterwegs sind. Die Applikation soll die Aufmerksamkeit an bekannten kritischen Stellen 
    entlang der Stammstrecke erhöhen und dadurch das Unfallrisiko reduzieren.

    \item[Motorradfahrende:] Fahrerinnen und Fahrer von Motorrädern, insbesondere auf Landstraßen und außerörtlichen 
    Strecken. Durch Warnhinweise vor unfallträchtigen Kurven und Streckenabschnitten soll das Risiko schwerer Unfälle 
    verringert werden.

    \item[Radfahrende und E-Bike-Nutzende:] Alltagsradlerinnen und -radler im urbanen Raum, die regelmäßig am Straßenverkehr 
    teilnehmen. Die Applikation unterstützt diese Zielgruppe bei der Identifikation und Umfahrung kritischer Kreuzungen 
    und unübersichtlicher Abschnitte.

    \item[Eltern / Erziehungsberechtigte:] Eltern, die sichere Wege (z.\,B. Schulwege) für ihre Kinder planen möchten. 
    Die Applikation kann zur Auswahl möglichst sicherer Routen und zur Kennzeichnung besonders kritischer Abschnitte 
    genutzt werden.
\end{description}

