\documentclass{report}

\usepackage[utf8]{inputenc} % Zeichenkodierung
\usepackage[ngerman]{babel} % Sprache (neue Rechtschreibung)

\usepackage{float}
\usepackage{placeins}
\usepackage[table,xcdraw]{xcolor} % Tabellenfarben
\usepackage{tabularx} % Dynamische Tabellenbreite
\usepackage{tcolorbox} % Graue Boxen
\usepackage{hyperref} % Links/URLs
\usepackage{todonotes} % TODO-Notizen
\usepackage{natbib} % Bibliographie
\usepackage{fancyhdr} % Header und Footer
\usepackage{multirow} % Mehrzeilige Tabellen
\usepackage{geometry} % Seitenlayout
\usepackage{color} % Textfarben
\usepackage{graphicx} % Grafiken (für \includegraphics)

% Seitenlayout
\geometry{
  bottom=3.5cm,
  headheight=20pt % realistisch, sonst meckert fancyhdr
}
\setlength{\headheight}{20pt}

% eigene Variablen
\newcommand{\semester}{Wintersemster 25/26}
\newcommand{\projektname}{Safe Kiel}


% Bibstyle
\bibliographystyle{plain}

% Header / Footer
\fancypagestyle{plain}{
  \fancyhf{}% Clear header/footer
  \fancyhead[R]{\includegraphics[width=4cm]{img/cau-logo-2017}} % rechter Header
  \fancyhead[L]{\leftmark} % linker Header
  \fancyfoot[R]{\thepage} % rechter Footer
  \fancyfoot[L] % linker Footer
}
\pagestyle{plain}

\renewcommand{\headrulewidth}{0.5pt}
\renewcommand{\footrulewidth}{0.2pt}
\renewcommand{\chaptermark}[1]{\markboth{{#1}}{}}

% Zahlen für Fußnoten
\renewcommand{\thefootnote}{\arabic{footnote}}
\renewcommand{\thempfootnote}{\arabic{mpfootnote}}

% Titelseite
\title{
  \vspace*{-3cm}
  Pflichtenheft\\[2mm]
  {\color{gray}
  \projektname\\
  \semester %Hier war gruppennamen definiert, was zu einem Error im Log führte (Patrick, 09.12.2025)
}\\[5mm]
  \includegraphics[width=\textwidth]{img/logo}
}

\author{
  \begin{tabular}{ll}
    Roni      & Aba \\
    Johannes  & Martinjonsan \\
    Patrick   & Wernowski \\
  \end{tabular}
}

\date{\today}

% Dokument
\begin{document}

  % erste Seiten ohne Seitenzahl
  \pagenumbering{gobble}
  \maketitle
  \tableofcontents

  % ab hier normale Nummerierung
  \clearpage
  \pagenumbering{arabic}

  \chapter{Produkteinsatz}\label{chp:produkteinsatz}
  \section{Anwendungsgebiet}
Die App KielSafe soll im Alltag der Verkehrsteilnehmenden helfen, Unfälle im Straßenverkehr zu vermeiden. 
Dazu werden vorhandene Unfalldaten ausgewertet, um Stellen mit einer auffälligen Häufung von Unfällen 
im Straßennetz von Kiel und der näheren Umgebung zu erkennen und auf einer Karte darzustellen. KielSafe wird als mobile Anwendung auf Smartphones genutzt. Anhand der aktuellen Position der Nutzenden per GPS wird geprüft, ob sie sich in der Nähe einer bekannten Unfallstelle befinden. Wenn dies der Fall ist, 
erhalten sie einen Hinweis direkt auf dem Mobiltelefon, zum Beispiel in Form einer kurzen Nachricht oder 
eines Hinweises auf der Karte. So sollen kritische Stellen bewusster wahrgenommen und die Aufmerksamkeit 
im Straßenverkehr erhöht werden. Die App versteht sich als Ergänzung zu bestehenden Maßnahmen der Verkehrssicherheitsarbeit. Sie ersetzt 
weder Verkehrszeichen noch bauliche Maßnahmen, sondern unterstützt die Nutzenden dabei, ihr eigenes 
Verhalten an bekannten Gefahrenschwerpunkten anzupassen und vorausschauender zu fahren.

\newpage

\section{Zielgruppe}
Die Applikation richtet sich generell an Verkehrsteilnehmende, die sich aktiv im Straßenverkehr bewegen und ein 
mobiles Endgerät mitführen. Im Fokus stehen dabei folgende Zielgruppen:

\begin{description}
    \item[Fahranfänger:] Personen mit geringer Fahrpraxis (z.\,B. Führerscheinneulinge, Fahrende im Rahmen von 
    begleitetem Fahren ab 17 Jahren). Diese Zielgruppe soll durch frühzeitige Warnungen vor Unfallhotspots dabei 
    unterstützt werden, Gefahrenschwerpunkte bewusster wahrzunehmen und sicherere Fahrstrategien zu entwickeln.

    \item[Berufspendler im Pkw:] Personen, die regelmäßig mit dem Auto auf wiederkehrenden Strecken (z.\,B. zwischen 
    Wohn- und Arbeitsort) unterwegs sind. Die Applikation soll die Aufmerksamkeit an bekannten kritischen Stellen 
    entlang der Stammstrecke erhöhen und dadurch das Unfallrisiko reduzieren.

    \item[Motorradfahrende:] Fahrerinnen und Fahrer von Motorrädern, insbesondere auf Landstraßen und außerörtlichen 
    Strecken. Durch Warnhinweise vor unfallträchtigen Kurven und Streckenabschnitten soll das Risiko schwerer Unfälle 
    verringert werden.

    \item[Radfahrende und E-Bike-Nutzende:] Alltagsradlerinnen und -radler im urbanen Raum, die regelmäßig am Straßenverkehr 
    teilnehmen. Die Applikation unterstützt diese Zielgruppe bei der Identifikation und Umfahrung kritischer Kreuzungen 
    und unübersichtlicher Abschnitte.

    \item[Eltern / Erziehungsberechtigte:] Eltern, die sichere Wege (z.\,B. Schulwege) für ihre Kinder planen möchten. 
    Die Applikation kann zur Auswahl möglichst sicherer Routen und zur Kennzeichnung besonders kritischer Abschnitte 
    genutzt werden.
\end{description}



    \chapter{Lizenz}\label{chp:lizenz}
    \pagenumbering{arabic} % Nummerierung starten
    \input{kapitel/lizenz}

  \chapter{Zielbestimmungen}\label{chp:zielbestimmungen}
  

    \begin{tcolorbox}
    Die Zielbestimmungen dienen dazu, die Anforderungen nach Priorität zu klassifizieren. Die Einteilung erfolgt in \textit{Muss}-, \textit{Soll}- und \textit{Kann}-Kriterien sowie \textit{Abgrenzungskriterien}, um den Projektumfang klar zu definieren.
    \end{tcolorbox}
    
    \section{Muss-Kriterien}
    Die folgenden Anforderungen sind essenziell für die Funktionsfähigkeit der Anwendung und müssen ohne Kompromisse umgesetzt werden:
    \begin{itemize}
        \item Das Smartphone notifiziert den User wenn er einem Unfall-Hotspot zu nahe kommt.
        \item Der User bestätigt einen Unfall-Hotspot
        \item Der User meldet unbekannte Unfall-Hotspots
        \item Implementierung einer Admin-Weboberfläche.
        \item Der User soll in der Lage sein Informationen an den Admin zu verschicken
        
    \end{itemize}
    
    \section{Soll-Kriterien}
    Diese Funktionen sind erwünscht und sollen nach Möglichkeit implementiert werden:
    \begin{itemize}
        \item Implementierung eines User-Login
        \item Die User sollen die Möglichkeit sich via Kommentare und Voting über die Unfall-Hotspots auszutauschen 
        \item Einzeichung einer Heat-Map in die Google Maps Maps Oberfläche.
     
    \end{itemize}
    
    \section{Kann-Kriterien}
    Diese Funktionen sind optional und werden nur umgesetzt, falls Zeit und Ressourcen es erlauben:
    \begin{itemize}
       \item Implementierung eines Forums.
    \end{itemize}
    
    \section{Abgrenzungskriterien}
    Folgende Anforderungen werden explizit nicht umgesetzt:
    \begin{itemize}
        \item Zahlungs- oder Monetarisierungsfunktionen innerhalb der Plattform.
    \end{itemize}


  \chapter{Produktfunktionen}\label{chp:produktfunktionen}
  
\section{Anwendungsfalldiagramm - Anwendung}

\begin{figure}[h]
	\centering
	
	\begin{tabularx}{\textwidth}{ p{.2\textwidth} | p{.2\textwidth} | X }
		\textbf{Akteur} & \textbf{Beschreibung} & \textbf{Verwendet in Anwendungsfall} \\ \hline
		Informatiker & Programmiert tolle Sachen & Programmieren, Kaffee trinken, Schlafen
	\end{tabularx}
	
	\caption{Beschreibung der Akteure}
	\label{fig:akteur-tabelle}
\end{figure}

\begin{figure}[h]
    \centering
    \includegraphics[width=\linewidth]{Doc/files/pflichtenheft/img/Anwendungsfalldiagramm.png}
    \caption{Anwendungsfalldiagramm - App}
    \label{fig:anwendungsfalldiagramm-app}
\end{figure}

\begin{figure}[h]
    \centering
    \begin{tabular}{l l}
        \textbf{M} & Muss-Feature \\
        \textbf{U} & Anwendungsfall für den User \\
        \textbf{A} & Anwendungsfall für den Admin \\
    \end{tabular}
    \caption{Legende}
    \label{fig:legende}
\end{figure}


\begin{figure}[h]
    \centering
    \begin{tabularx}{\textwidth}{X|X}
        \textbf{Anwendungsfall ID} & MU-1 \\ \hline
        \textbf{Anwendungsfallname} & Location als false positive Melden\\ \hline
        \textbf{Initiierender Akteur} & Anwender \\ \hline
        \textbf{Weitere Akteure} & - \\ \hline
        \textbf{Kurzbeschreibung} & 
        Der Anwender hat die GPS-Daten eines Unfall-Hotspotes entdeckt, welche nicht der Realität korrektheit. \\ \hline
        \textbf{Vorbedingungen} & 
        \\ \hline
        \textbf{Nachbedingungen} & 
        Der Request wurde an den Admin gemeldet. \\ \hline
        \textbf{Ablauf} &
            \begin{enumerate}
                \item Der Anwender befindet sich auf der Startseite der APP und klickt auf den Button "Melden".
                \item Das System bietet den Nutzer den aktuellen Standort als false Positive zu melden oder eine GPS-Location selber zu wählen.
            \end{enumerate}
        \\ \hline
        \textbf{Alternative} & - \\ \hline
        \textbf{Ausnahme} & - \\ \hline
        \textbf{Benutzte Anwendungsfälle} & - \\ \hline
        \textbf{Spezielle Anforderungen} & - \\ \hline
        \textbf{Annahmen} & - 
    \end{tabularx}
    \caption{Anwendungsfall MU-1 (Location als false positive Melden)}
    \label{fig:anwendungsfall-app-tabelle-mz-1}
\end{figure}
\begin{figure}[h]
    \centering
    \begin{tabularx}{\textwidth}{X|X}
        \textbf{Anwendungsfall ID} & MU-2 \\ \hline
        \textbf{Anwendungsfallname} & Sich einem Unfallhotspot nähern  \\ \hline
        \textbf{Initiierender Akteur} & Anwender \\ \hline
        \textbf{Weitere Akteure} & - \\ \hline
        \textbf{Kurzbeschreibung} & 
        Der Anwender nähert sich einem Hotspot und wird vom System gewarnt. \\ \hline
        \textbf{Vorbedingungen} & 
         \\ \hline
        \textbf{Nachbedingungen} & 
        Der Anwender wurde gewarnt \\ \hline
        \textbf{Ablauf} &
            \begin{enumerate}
                \item Der Anwender nähert sich einem Unfall Hotspot
                \item Das System warnt den Anwwender via gewünschten Notification
        
            \end{enumerate}
        \\ \hline
        \textbf{Alternative} & - \\ \hline
        \textbf{Ausnahme} & - \\ \hline
        \textbf{Benutzte Anwendungsfälle} &-  \\ \hline
        \textbf{Spezielle Anforderungen} & - \\ \hline
        \textbf{Annahmen} & - 
    \end{tabularx}
    \caption{Anwendungsfall MU-2 (Sich einem Unfallhotspot nähern)}
    \label{fig:anwendungsfall-app-tabelle-mu-2}
\end{figure}
\begin{figure}[h]
    \centering
    \begin{tabularx}{\textwidth}{X|X}
        \textbf{Anwendungsfall ID} & MU-3 \\ \hline
        \textbf{Anwendungsfallname} & Anwender bestätigt einen Unfall-Hotspot \\ \hline
        \textbf{Initiierender Akteur} & Anwender \\ \hline
        \textbf{Weitere Akteure} & / \\ \hline
        \textbf{Kurzbeschreibung} & 
        Der Anwender wurde vom System gewarnt und bestätigt nun die Gefahrenstelle. \\ \hline
        \textbf{Vorbedingungen} & 
        Der Anwender kam einen Unfallhotspot zu nahe (MU-2). \\ \hline
        \textbf{Nachbedingungen} & 
        Das System enthält vom Benutzer die Bestätigung und vermerkt es. \\ \hline
        \textbf{Ablauf} &
            \begin{enumerate}
                \item (Use-Case „Sich einem Unfallhotspot nähern“) muss zuvor erfolgt sein.
                \item Der Anwender bestätigt den Hotspot via Shortcut oder Auduio
                \item Das System vermerkt die Bestätigung.
            \end{enumerate}
        \\ \hline
        \textbf{Alternative} & / \\ \hline
        \textbf{Ausnahme} & / \\ \hline
        \textbf{Benutzte Anwendungsfälle} & Sich einem Unfallhotspot nähern \\ \hline
        \textbf{Spezielle Anforderungen} & / \\ \hline
        \textbf{Annahmen} & / 
    \end{tabularx}
    \caption{Anwendungsfall MU-3 (Anwender bestätigt einen Unfall-Hotspot)}
    \label{fig:anwendungsfall-app-tabelle-mu-3}
\end{figure}
\begin{figure}[h]
    \centering
    \begin{tabularx}{\textwidth}{X|X}
        \textbf{Anwendungsfall ID} & MU-4 \\ \hline
        \textbf{Anwendungsfallname} & Unfallhotspot hinzufügen \\ \hline
        \textbf{Initiierender Akteur} & Anwender \\ \hline
        \textbf{Weitere Akteure} & / \\ \hline
        \textbf{Kurzbeschreibung} & 
        Der Anwender fügt einen Unfallhotspot in die Datenbank ein. \\ \hline
        \textbf{Vorbedingungen} & 
        Der Anwender hat einen nocht nicht in die Datenbank eingepflegten Hotspot lokalisiert. \\ \hline
        \textbf{Nachbedingungen} & 
        Der Anwender hat den Hotspot in die Datenbank eingepleft. \\ \hline
        \textbf{Ablauf} &
            \begin{enumerate}
                \item Der Anwender ist an einer gefährlichen Stelle im Straßenverkehr
                \item Der Anwender klickt auf der App den Button „Add Hotspot“.
                \item Das System fragt den Anwender ob er seinen Live-Standort einpflegen will oder manuell GPS-Daten eingeben will.
                \item Der Anwender verwendet seinen Live-Standort zur Einpflege
                \item Das System fügt den Hotspot in die Datenbank hinzu.
            \end{enumerate}
        \\ \hline
        \textbf{Alternative} &  \begin{enumerate}
                \item Der Anwender ist an einer gefährlichen Stelle im Straßenverkehr.
                \item Der Anwender klickt auf der App den Button „Add Hotspot“.
                \item Das System fragt den Anwender ob er seinen Live-Standort einpflegen will oder manuell GPS-Daten eingeben will.
                \item Der Anwender tippt manuell die GPS-Daten ein.
                \item Das System fügt den Hotspot in die Datenbank hinzu.
            \end{enumerate}
            \\ \hline 
        \textbf{Ausnahme} & / \\ \hline
        \textbf{Benutzte Anwendungsfälle} & -\\ \hline
        \textbf{Spezielle Anforderungen} & / \\ \hline
        \textbf{Annahmen} & /
    \end{tabularx}
    \caption{Anwendungsfall MU-4 (Unfallhotspot hinzufügen)}
    \label{fig:anwendungsfall-app-tabelle-mu-4-Unfallhotspot hinzufügen}
\end{figure}
%\input{kapitel/anwendungsfaelle/mz5}
%\input{kapitel/anwendungsfaelle/mz6}
%\input{kapitel/anwendungsfaelle/mz7}
%\input{kapitel/anwendungsfaelle/mz8}
%\input{kapitel/anwendungsfaelle/mn1} 
%\input{kapitel/anwendungsfaelle/mn2}
%\input{kapitel/anwendungsfaelle/mn3}
\begin{figure}[h]
    \centering
    \begin{tabularx}{\textwidth}{X|X}
        \textbf{Anwendungsfall ID} & MA-1 \\ \hline
        \textbf{Anwendungsfallname} & Admin Login \\ \hline
        \textbf{Initiierender Akteur} & Admin \\ \hline
        \textbf{Weitere Akteure} & / \\ \hline
        \textbf{Kurzbeschreibung} & 
        Der Admin loggt sich mit seinen spezifischen Daten ein. \\ \hline
        \textbf{Vorbedingungen} & 
        Admin-Zugang existiert. \\ \hline
        \textbf{Nachbedingungen} & 
        Admin ist authentifiziert und kann Admin-Funktionen nutzen. \\ \hline
        \textbf{Ablauf} &
            \begin{enumerate}
                \item Admin klickt auf „Admin Login“.
                \item Admin gibt Zugangsdaten ein.
                \item System prüft die Rechte.
                \item Admin erhält spezifische Rechte.
            \end{enumerate}
        \\ \hline
        \textbf{Alternative} &
            \begin{enumerate}
                \item Falsche Login-Daten: Fehlermeldung.
            \end{enumerate}
        \\ \hline
        \textbf{Ausnahme} & / \\ \hline
        \textbf{Benutzte Anwendungsfälle} & / \\ \hline
        \textbf{Spezielle Anforderungen} & / \\ \hline
        \textbf{Annahmen} & /
    \end{tabularx}
    \caption{Anwendungsfall MA-1 (Admin Login)}
    \label{fig:anwendungsfall-app-tabelle-ma-1}
\end{figure}
\begin{figure}[h]
    \centering
    \begin{tabularx}{\textwidth}{X|X}
        \textbf{Anwendungsfall ID} & MA-2 \\ \hline
        \textbf{Anwendungsfallname} & GPS-Daten verwalten \\ \hline
        \textbf{Initiierender Akteur} & Admin \\ \hline
        \textbf{Weitere Akteure} & / \\ \hline
        \textbf{Kurzbeschreibung} & 
        Der Admin verwaltet über seine Web-Oberfläche die GPS-Daten. Hier wählt er ganz spezifisch GPS-Daten aus Requests aus oder er verwaltet nach eigenem Ermessen \\ \hline
        \textbf{Vorbedingungen} & 
        Admin ist eingeloggt. \\ \hline
        \textbf{Nachbedingungen} & 
        Die GPS-Daten wurden von dem Admin bearbeitet.\\ \hline
        \textbf{Ablauf} &
            \begin{enumerate}
                \item (Use-Case „Admin Login“) muss zuvor erfolgt sein.
                \item Admin öffnet „Einstellungen“.
                \item Admin wählt „GPS-Daten verwalten“.
                \item Admin ändert Parameter.
                \item System übernimmt die Parameter.
            \end{enumerate}
        \\ \hline
        \textbf{Alternative} & / \\ \hline
        \textbf{Ausnahme} & / \\ \hline
        \textbf{Benutzte Anwendungsfälle} & Admin Login \\ \hline
        \textbf{Spezielle Anforderungen} & / \\ \hline
        \textbf{Annahmen} & /
    \end{tabularx}
    \caption{Anwendungsfall MA-2 (GPS-Daten verwalten)}
    \label{fig:anwendungsfall-app-tabelle-ma-2}
\end{figure}
\begin{figure}[h]
    \centering
    \begin{tabularx}{\textwidth}{X|X}
        \textbf{Anwendungsfall ID} & MA-3 \\ \hline
        \textbf{Anwendungsfallname} & Nutzer verwalten \\ \hline
        \textbf{Initiierender Akteur} & Admin \\ \hline
        \textbf{Weitere Akteure} & / \\ \hline
        \textbf{Kurzbeschreibung} & 
        Der Admin kann Nutzer hinzufügen oder entfernen. \\ \hline
        \textbf{Vorbedingungen} & 
        Der Admin hat sich als Admin eingeloggt. \\ \hline
        \textbf{Nachbedingungen} & 
        Nutzerkonten sind entsprechend angepasst. \\ \hline
        \textbf{Ablauf} &
            \begin{enumerate}
                \item (Use-Case „Admin Login“) muss zuvor erfolgt sein.
                \item Admin öffnet Menüpunkt „Nutzer verwalten“.
                \item Admin legt neue Nutzer an oder ändert/löscht bestehende Nutzer.
            \end{enumerate}
        \\ \hline
        \textbf{Alternative} & / \\ \hline
        \textbf{Ausnahme} & / \\ \hline
        \textbf{Benutzte Anwendungsfälle} & Admin Login \\ \hline
        \textbf{Spezielle Anforderungen} & / \\ \hline
        \textbf{Annahmen} & /
    \end{tabularx}
    \caption{Anwendungsfall MA-3 (Nutzer verwalten)}
    \label{fig:anwendungsfall-app-tabelle-ma-3}
\end{figure}
\begin{figure}[h]
    \centering
    \begin{tabularx}{\textwidth}{X|X}
        \textbf{Anwendungsfall ID} & MA-4 \\ \hline
        \textbf{Anwendungsfallname} & Anfragen verwalten \\ \hline
        \textbf{Initiierender Akteur} & Admin \\ \hline
        \textbf{Weitere Akteure} & Nutzer \\ \hline
        \textbf{Kurzbeschreibung} & 
        Der Admin sieht alle offenen, vom Nutzer gestellten Anfragen und entscheidet darüber. \\ \hline
        \textbf{Vorbedingungen} & 
        Admin ist eingeloggt. \\ \hline
        \textbf{Nachbedingungen} & 
        Die betroffene Anfrage ist akzeptiert oder abgelehnt; das System ist ggf. angepasst. \\ \hline
        \textbf{Ablauf} &
            \begin{enumerate}
                \item (Use-Case „Admin Login“) muss zuvor erfolgt sein.
                \item Admin öffnet „Anfragen verwalten“.
                \item Admin genehmigt oder lehnt einzelne Anfragen ab.
                \item System ändert ggf. die Daten.
            \end{enumerate}
        \\ \hline
        \textbf{Alternative} & 
        Keine Anfragen vorhanden: System bleibt auf dem aktuellen Stand. \\ \hline
        \textbf{Ausnahme} & / \\ \hline
        \textbf{Benutzte Anwendungsfälle} & Admin Login \\ \hline
        \textbf{Spezielle Anforderungen} & 
        Es sollte mindestens eine Anfrage vom Nutzer existieren. \\ \hline
        \textbf{Annahmen} & /
    \end{tabularx}
    \caption{Anwendungsfall MA-4 (Anfragen verwalten)}
    \label{fig:anwendungsfall-app-tabelle-ma-4}
\end{figure}
%\input{kapitel/anwendungsfaelle/ma5} Fehlt oder ist falsch? (Patrick,09.12.2025)


  

  \chapter{Glossar}\label{chp:glossar}
  \begin{table}[h]
	\centering
	\begin{tabularx}{\textwidth}{>{\hsize=0.4\hsize}X >{\hsize=1.6\hsize}X}
		\rowcolor[HTML]{C0C0C0} 
		\textbf{Abkürzung} & \textbf{Beschreibung} \\
		/ & /. \\


	\end{tabularx}
	\caption{Glossar}
	\label{table:glossar}
\end{table}


  \bibliography{references}

  % falls du am Ende wieder keine Seitenzahlen willst:
  % \pagenumbering{gobble}

\end{document}